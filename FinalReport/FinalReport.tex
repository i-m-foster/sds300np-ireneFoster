%  LaTeX support: latex@mdpi.com
%  In case you need support, please attach all files that are necessary for compiling as well as the log file, and specify the details of your LaTeX setup (which operating system and LaTeX version / tools you are using).

%=================================================================
\documentclass[,article,submit,moreauthors,pdftex]{mdpi}

% If you would like to post an early version of this manuscript as a preprint, you may use preprint as the journal and change 'submit' to 'accept'. The document class line would be, e.g., \documentclass[preprints,article,accept,moreauthors,pdftex]{mdpi}. This is especially recommended for submission to arXiv, where line numbers should be removed before posting. For preprints.org, the editorial staff will make this change immediately prior to posting.

%% Some pieces required from the pandoc template
\setlist[itemize]{leftmargin=*,labelsep=5.8mm}
\setlist[enumerate]{leftmargin=*,labelsep=4.9mm}


%--------------------
% Class Options:
%--------------------
%----------
% journal
%----------
% Choose between the following MDPI journals:
% acoustics, actuators, addictions, admsci, aerospace, agriculture, agriengineering, agronomy, algorithms, animals, antibiotics, antibodies, antioxidants, applsci, arts, asc, asi, atmosphere, atoms, axioms, batteries, bdcc, behavsci , beverages, bioengineering, biology, biomedicines, biomimetics, biomolecules, biosensors, brainsci , buildings, cancers, carbon , catalysts, cells, ceramics, challenges, chemengineering, chemistry, chemosensors, children, cleantechnol, climate, clockssleep, cmd, coatings, colloids, computation, computers, condensedmatter, cosmetics, cryptography, crystals, dairy, data, dentistry, designs , diagnostics, diseases, diversity, drones, econometrics, economies, education, electrochem, electronics, energies, entropy, environments, epigenomes, est, fermentation, fibers, fire, fishes, fluids, foods, forecasting, forests, fractalfract, futureinternet, futurephys, galaxies, games, gastrointestdisord, gels, genealogy, genes, geohazards, geosciences, geriatrics, hazardousmatters, healthcare, heritage, highthroughput, horticulturae, humanities, hydrology, ijerph, ijfs, ijgi, ijms, ijns, ijtpp, informatics, information, infrastructures, inorganics, insects, instruments, inventions, iot, j, jcdd, jcm, jcp, jcs, jdb, jfb, jfmk, jimaging, jintelligence, jlpea, jmmp, jmse, jnt, jof, joitmc, jpm, jrfm, jsan, land, languages, laws, life, literature, logistics, lubricants, machines, magnetochemistry, make, marinedrugs, materials, mathematics, mca, medicina, medicines, medsci, membranes, metabolites, metals, microarrays, micromachines, microorganisms, minerals, modelling, molbank, molecules, mps, mti, nanomaterials, ncrna, neuroglia, nitrogen, notspecified, nutrients, ohbm, particles, pathogens, pharmaceuticals, pharmaceutics, pharmacy, philosophies, photonics, physics, plants, plasma, polymers, polysaccharides, preprints , proceedings, processes, proteomes, psych, publications, quantumrep, quaternary, qubs, reactions, recycling, religions, remotesensing, reports, resources, risks, robotics, safety, sci, scipharm, sensors, separations, sexes, signals, sinusitis, smartcities, sna, societies, socsci, soilsystems, sports, standards, stats, surfaces, surgeries, sustainability, symmetry, systems, technologies, test, toxics, toxins, tropicalmed, universe, urbansci, vaccines, vehicles, vetsci, vibration, viruses, vision, water, wem, wevj

%---------
% article
%---------
% The default type of manuscript is "article", but can be replaced by:
% abstract, addendum, article, benchmark, book, bookreview, briefreport, casereport, changes, comment, commentary, communication, conceptpaper, conferenceproceedings, correction, conferencereport, expressionofconcern, extendedabstract, meetingreport, creative, datadescriptor, discussion, editorial, essay, erratum, hypothesis, interestingimages, letter, meetingreport, newbookreceived, obituary, opinion, projectreport, reply, retraction, review, perspective, protocol, shortnote, supfile, technicalnote, viewpoint
% supfile = supplementary materials

%----------
% submit
%----------
% The class option "submit" will be changed to "accept" by the Editorial Office when the paper is accepted. This will only make changes to the frontpage (e.g., the logo of the journal will get visible), the headings, and the copyright information. Also, line numbering will be removed. Journal info and pagination for accepted papers will also be assigned by the Editorial Office.

%------------------
% moreauthors
%------------------
% If there is only one author the class option oneauthor should be used. Otherwise use the class option moreauthors.

%---------
% pdftex
%---------
% The option pdftex is for use with pdfLaTeX. If eps figures are used, remove the option pdftex and use LaTeX and dvi2pdf.

%=================================================================
\firstpage{1}
\makeatletter
\setcounter{page}{\@firstpage}
\makeatother
\pubvolume{xx}
\issuenum{1}
\articlenumber{5}
\pubyear{2019}
\copyrightyear{2019}
%\externaleditor{Academic Editor: name}
\history{Received: date; Accepted: date; Published: date}
\updates{yes} % If there is an update available, un-comment this line

%% MDPI internal command: uncomment if new journal that already uses continuous page numbers
%\continuouspages{yes}

%------------------------------------------------------------------
% The following line should be uncommented if the LaTeX file is uploaded to arXiv.org
%\pdfoutput=1

%=================================================================
% Add packages and commands here. The following packages are loaded in our class file: fontenc, calc, indentfirst, fancyhdr, graphicx, lastpage, ifthen, lineno, float, amsmath, setspace, enumitem, mathpazo, booktabs, titlesec, etoolbox, amsthm, hyphenat, natbib, hyperref, footmisc, geometry, caption, url, mdframed, tabto, soul, multirow, microtype, tikz

%=================================================================
%% Please use the following mathematics environments: Theorem, Lemma, Corollary, Proposition, Characterization, Property, Problem, Example, ExamplesandDefinitions, Hypothesis, Remark, Definition
%% For proofs, please use the proof environment (the amsthm package is loaded by the MDPI class).

%=================================================================
% Full title of the paper (Capitalized)
\Title{Full title of the paper (Capitalized)}

% Authors, for the paper (add full first names)
\Author{Irene
Foster$^{1,*}$\href{https://orcid.org/0000-0001-9681-4786}{\orcidicon}}

% Authors, for metadata in PDF
\AuthorNames{Irene Foster}

% Affiliations / Addresses (Add [1] after \address if there is only one affiliation.)
\address{%
$^{1}$ \quad Smith College - Department of Statistical \& Data Sciences
Northampton, MA, USA; \\
}
% Contact information of the corresponding author
\corres{Correspondence: \href{mailto:ifoster25@smith.edu}{\nolinkurl{ifoster25@smith.edu}}}

% Current address and/or shared authorship








% The commands \thirdnote{} till \eighthnote{} are available for further notes

% Simple summary

% Abstract (Do not insert blank lines, i.e. \\)
\abstract{{[}Usually, 150-200 words overview of the research paper you
have conducted -- includes short introduction to the research question,
review to methods, review of key results, conclusion for the results in
line with the research paper.{]}}

% Keywords
\keyword{ADHD; behavioral activity; machine learning; time series}

% The fields PACS, MSC, and JEL may be left empty or commented out if not applicable
%\PACS{J0101}
%\MSC{}
%\JEL{}

%%%%%%%%%%%%%%%%%%%%%%%%%%%%%%%%%%%%%%%%%%
% Only for the journal Diversity
%\LSID{\url{http://}}

%%%%%%%%%%%%%%%%%%%%%%%%%%%%%%%%%%%%%%%%%%
% Only for the journal Applied Sciences:
%\featuredapplication{Authors are encouraged to provide a concise description of the specific application or a potential application of the work. This section is not mandatory.}
%%%%%%%%%%%%%%%%%%%%%%%%%%%%%%%%%%%%%%%%%%

%%%%%%%%%%%%%%%%%%%%%%%%%%%%%%%%%%%%%%%%%%
% Only for the journal Data:
%\dataset{DOI number or link to the deposited data set in cases where the data set is published or set to be published separately. If the data set is submitted and will be published as a supplement to this paper in the journal Data, this field will be filled by the editors of the journal. In this case, please make sure to submit the data set as a supplement when entering your manuscript into our manuscript editorial system.}

%\datasetlicense{license under which the data set is made available (CC0, CC-BY, CC-BY-SA, CC-BY-NC, etc.)}

%%%%%%%%%%%%%%%%%%%%%%%%%%%%%%%%%%%%%%%%%%
% Only for the journal Toxins
%\keycontribution{The breakthroughs or highlights of the manuscript. Authors can write one or two sentences to describe the most important part of the paper.}

%\setcounter{secnumdepth}{4}
%%%%%%%%%%%%%%%%%%%%%%%%%%%%%%%%%%%%%%%%%%


% tightlist command for lists without linebreak
\providecommand{\tightlist}{%
  \setlength{\itemsep}{0pt}\setlength{\parskip}{0pt}}




\begin{document}


%%%%%%%%%%%%%%%%%%%%%%%%%%%%%%%%%%%%%%%%%%

\hypertarget{introduction}{%
\section{Introduction}\label{introduction}}

\emph{ADD}

\hypertarget{methods}{%
\section{Methods}\label{methods}}

\hypertarget{dataset}{%
\subsection{Dataset}\label{dataset}}

{[}Contrast this to what would have been your ideal dataset. How did you
arrive at choosing this dataset. Provide details on the dataset, size,
gender split and count, age, how it was collected, ethical concerns,
etc. Also provide details on data collection and access. Usually 100-200
words.{]}

\hypertarget{data-cleaning}{%
\subsection{Data Cleaning}\label{data-cleaning}}

{[}Comment on any data cleaning considerations. Varies in length.{]}

\hypertarget{variables-of-interest}{%
\subsection{Variables of Interest}\label{variables-of-interest}}

{[}Describe the variables that you are using in your datasets. Describe
the rationale for using these variables. Varies in length.{]}

\hypertarget{other-important-features}{%
\subsection{Other Important Features}\label{other-important-features}}

{[}Also comment on other things which are relevant to your research
paper. Varies in length.{]}

\#\#Data Analyses

{[}What analyses are you doing why. Varies in length.{]}

\hypertarget{results}{%
\section{Results}\label{results}}

{[}Summary of results analyses 1{]}

{[}Add visualizations from analyses. Varies in length.{]}

{[}Summary of results analyses 2{]}

{[}Add visualizations from analyses. Varies in length.{]}

\hypertarget{discussion}{%
\section{Discussion}\label{discussion}}

{[}Explain what your results mean in the context of the literature cited
in the introduction. Minimum of 750 words.{]}

\emph{potential issue - diagnosis is still coming from people}

\hypertarget{future-directions}{%
\section{Future Directions}\label{future-directions}}

{[}How could someone continue the work? Around 150-200 words.{]}

\hypertarget{personal-reflection}{%
\section{Personal Reflection}\label{personal-reflection}}

{[}What was your original research question? Why did you have to
deviate? What ideal dataset would be recommend collecting? How was this
process of developing a research paper? 250 words{]}

\hypertarget{code-availability}{%
\section{Code availability}\label{code-availability}}

All analysis code for this article is available at:
\url{https://github.com/i-m-foster/sds300np-ireneFoster}

\hypertarget{acknowledgements}{%
\section{Acknowledgements}\label{acknowledgements}}

{[}Mention who you would like to thank. Any grants or people. Varies in
length.{]}

Bulleted lists look like this:

\begin{itemize}
\tightlist
\item
  First bullet
\item
  Second bullet
\item
  Third bullet
\end{itemize}

Numbered lists can be added as follows:

\begin{enumerate}
\def\labelenumi{\arabic{enumi}.}
\tightlist
\item
  First item
\item
  Second item
\item
  Third item
\end{enumerate}

The text continues here.

All figures and tables should be cited in the main text as Figure 1,
Table 1, etc.

\begin{figure}[H]
\centering
\includegraphics[width=3 cm]{logo-mdpi}
\caption{This is a figure, Schemes follow the same formatting. If there are multiple panels, they should be listed as: (\textbf{a}) Description of what is contained in the first panel. (\textbf{b}) Description of what is contained in the second panel. Figures should be placed in the main text near to the first time they are cited. A caption on a single line should be centered.}
\end{figure}

\begin{table}[H]
\caption{This is a table caption. Tables should be placed in the main text near to the first time they are cited.}
\centering
%% \tablesize{} %% You can specify the fontsize here, e.g.  \tablesize{\footnotesize}. If commented out \small will be used.
\begin{tabular}{ccc}
\toprule
\textbf{Title 1}    & \textbf{Title 2}  & \textbf{Title 3}\\
\midrule
entry 1     & data          & data\\
entry 2     & data          & data\\
\bottomrule
\end{tabular}
\end{table}

% %%%%%%%%%%%%%%%%%%%%%%%%%%%%%%%%%%%%%%%%%%
% %% optional
% \supplementary{The following are available online at www.mdpi.com/link, Figure S1: title, Table S1: title, Video S1: title.}
%
% % Only for the journal Methods and Protocols:
% % If you wish to submit a video article, please do so with any other supplementary material.
% % \supplementary{The following are available at www.mdpi.com/link: Figure S1: title, Table S1: title, Video S1: title. A supporting video article is available at doi: link.}

\vspace{6pt}

%%%%%%%%%%%%%%%%%%%%%%%%%%%%%%%%%%%%%%%%%%

%%%%%%%%%%%%%%%%%%%%%%%%%%%%%%%%%%%%%%%%%%

%%%%%%%%%%%%%%%%%%%%%%%%%%%%%%%%%%%%%%%%%%
\conflictsofinterest{The authors declare no conflict of interest.}

%%%%%%%%%%%%%%%%%%%%%%%%%%%%%%%%%%%%%%%%%%
%% optional


%%%%%%%%%%%%%%%%%%%%%%%%%%%%%%%%%%%%%%%%%%
% Citations and References in Supplementary files are permitted provided that they also appear in the reference list here.

%=====================================
% References, variant A: internal bibliography
%=====================================
%\reftitle{References}
%\begin{thebibliography}{999}
% Reference 1
%\bibitem[Author1(year)]{ref-journal}
%Author1, T. The title of the cited article. {\em Journal Abbreviation} {\bf 2008}, {\em 10}, 142--149.
% Reference 2
%\bibitem[Author2(year)]{ref-book}
%Author2, L. The title of the cited contribution. In {\em The Book Title}; Editor1, F., Editor2, A., Eds.; Publishing House: City, Country, 2007; pp. 32--58.
%\end{thebibliography}

% The following MDPI journals use author-date citation: Arts, Econometrics, Economies, Genealogy, Humanities, IJFS, JRFM, Laws, Religions, Risks, Social Sciences. For those journals, please follow the formatting guidelines on http://www.mdpi.com/authors/references
% To cite two works by the same author: \citeauthor{ref-journal-1a} (\citeyear{ref-journal-1a}, \citeyear{ref-journal-1b}). This produces: Whittaker (1967, 1975)
% To cite two works by the same author with specific pages: \citeauthor{ref-journal-3a} (\citeyear{ref-journal-3a}, p. 328; \citeyear{ref-journal-3b}, p.475). This produces: Wong (1999, p. 328; 2000, p. 475)

%=====================================
% References, variant B: external bibliography
%=====================================
\reftitle{References}
\externalbibliography{yes}
\bibliography{mybibfile.bib}

%%%%%%%%%%%%%%%%%%%%%%%%%%%%%%%%%%%%%%%%%%
%% optional

%% for journal Sci
%\reviewreports{\\
%Reviewer 1 comments and authors’ response\\
%Reviewer 2 comments and authors’ response\\
%Reviewer 3 comments and authors’ response
%}

%%%%%%%%%%%%%%%%%%%%%%%%%%%%%%%%%%%%%%%%%%


\end{document}
