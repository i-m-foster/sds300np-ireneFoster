%  LaTeX support: latex@mdpi.com
%  In case you need support, please attach all files that are necessary for compiling as well as the log file, and specify the details of your LaTeX setup (which operating system and LaTeX version / tools you are using).

%=================================================================
\documentclass[,article,submit,moreauthors,pdftex]{mdpi}

% If you would like to post an early version of this manuscript as a preprint, you may use preprint as the journal and change 'submit' to 'accept'. The document class line would be, e.g., \documentclass[preprints,article,accept,moreauthors,pdftex]{mdpi}. This is especially recommended for submission to arXiv, where line numbers should be removed before posting. For preprints.org, the editorial staff will make this change immediately prior to posting.

%% Some pieces required from the pandoc template
\setlist[itemize]{leftmargin=*,labelsep=5.8mm}
\setlist[enumerate]{leftmargin=*,labelsep=4.9mm}


%--------------------
% Class Options:
%--------------------
%----------
% journal
%----------
% Choose between the following MDPI journals:
% acoustics, actuators, addictions, admsci, aerospace, agriculture, agriengineering, agronomy, algorithms, animals, antibiotics, antibodies, antioxidants, applsci, arts, asc, asi, atmosphere, atoms, axioms, batteries, bdcc, behavsci , beverages, bioengineering, biology, biomedicines, biomimetics, biomolecules, biosensors, brainsci , buildings, cancers, carbon , catalysts, cells, ceramics, challenges, chemengineering, chemistry, chemosensors, children, cleantechnol, climate, clockssleep, cmd, coatings, colloids, computation, computers, condensedmatter, cosmetics, cryptography, crystals, dairy, data, dentistry, designs , diagnostics, diseases, diversity, drones, econometrics, economies, education, electrochem, electronics, energies, entropy, environments, epigenomes, est, fermentation, fibers, fire, fishes, fluids, foods, forecasting, forests, fractalfract, futureinternet, futurephys, galaxies, games, gastrointestdisord, gels, genealogy, genes, geohazards, geosciences, geriatrics, hazardousmatters, healthcare, heritage, highthroughput, horticulturae, humanities, hydrology, ijerph, ijfs, ijgi, ijms, ijns, ijtpp, informatics, information, infrastructures, inorganics, insects, instruments, inventions, iot, j, jcdd, jcm, jcp, jcs, jdb, jfb, jfmk, jimaging, jintelligence, jlpea, jmmp, jmse, jnt, jof, joitmc, jpm, jrfm, jsan, land, languages, laws, life, literature, logistics, lubricants, machines, magnetochemistry, make, marinedrugs, materials, mathematics, mca, medicina, medicines, medsci, membranes, metabolites, metals, microarrays, micromachines, microorganisms, minerals, modelling, molbank, molecules, mps, mti, nanomaterials, ncrna, neuroglia, nitrogen, notspecified, nutrients, ohbm, particles, pathogens, pharmaceuticals, pharmaceutics, pharmacy, philosophies, photonics, physics, plants, plasma, polymers, polysaccharides, preprints , proceedings, processes, proteomes, psych, publications, quantumrep, quaternary, qubs, reactions, recycling, religions, remotesensing, reports, resources, risks, robotics, safety, sci, scipharm, sensors, separations, sexes, signals, sinusitis, smartcities, sna, societies, socsci, soilsystems, sports, standards, stats, surfaces, surgeries, sustainability, symmetry, systems, technologies, test, toxics, toxins, tropicalmed, universe, urbansci, vaccines, vehicles, vetsci, vibration, viruses, vision, water, wem, wevj

%---------
% article
%---------
% The default type of manuscript is "article", but can be replaced by:
% abstract, addendum, article, benchmark, book, bookreview, briefreport, casereport, changes, comment, commentary, communication, conceptpaper, conferenceproceedings, correction, conferencereport, expressionofconcern, extendedabstract, meetingreport, creative, datadescriptor, discussion, editorial, essay, erratum, hypothesis, interestingimages, letter, meetingreport, newbookreceived, obituary, opinion, projectreport, reply, retraction, review, perspective, protocol, shortnote, supfile, technicalnote, viewpoint
% supfile = supplementary materials

%----------
% submit
%----------
% The class option "submit" will be changed to "accept" by the Editorial Office when the paper is accepted. This will only make changes to the frontpage (e.g., the logo of the journal will get visible), the headings, and the copyright information. Also, line numbering will be removed. Journal info and pagination for accepted papers will also be assigned by the Editorial Office.

%------------------
% moreauthors
%------------------
% If there is only one author the class option oneauthor should be used. Otherwise use the class option moreauthors.

%---------
% pdftex
%---------
% The option pdftex is for use with pdfLaTeX. If eps figures are used, remove the option pdftex and use LaTeX and dvi2pdf.

%=================================================================
\firstpage{1}
\makeatletter
\setcounter{page}{\@firstpage}
\makeatother
\pubvolume{xx}
\issuenum{1}
\articlenumber{5}
\pubyear{2019}
\copyrightyear{2019}
%\externaleditor{Academic Editor: name}
\history{Received: date; Accepted: date; Published: date}
\updates{yes} % If there is an update available, un-comment this line

%% MDPI internal command: uncomment if new journal that already uses continuous page numbers
%\continuouspages{yes}

%------------------------------------------------------------------
% The following line should be uncommented if the LaTeX file is uploaded to arXiv.org
%\pdfoutput=1

%=================================================================
% Add packages and commands here. The following packages are loaded in our class file: fontenc, calc, indentfirst, fancyhdr, graphicx, lastpage, ifthen, lineno, float, amsmath, setspace, enumitem, mathpazo, booktabs, titlesec, etoolbox, amsthm, hyphenat, natbib, hyperref, footmisc, geometry, caption, url, mdframed, tabto, soul, multirow, microtype, tikz

%=================================================================
%% Please use the following mathematics environments: Theorem, Lemma, Corollary, Proposition, Characterization, Property, Problem, Example, ExamplesandDefinitions, Hypothesis, Remark, Definition
%% For proofs, please use the proof environment (the amsthm package is loaded by the MDPI class).

%=================================================================
% Full title of the paper (Capitalized)
\Title{Predicting ADHD Using Activity Time Series Data}

% Authors, for the paper (add full first names)
\Author{Irene
Foster$^{1,*}$\href{https://orcid.org/0000-0001-9681-4786}{\orcidicon}}

% Authors, for metadata in PDF
\AuthorNames{Irene Foster}

% Affiliations / Addresses (Add [1] after \address if there is only one affiliation.)
\address{%
$^{1}$ \quad Smith College - Department of Statistical \& Data Sciences
Northampton, MA, USA; \\
}
% Contact information of the corresponding author
\corres{Correspondence: \href{mailto:ifoster25@smith.edu}{\nolinkurl{ifoster25@smith.edu}}}

% Current address and/or shared authorship








% The commands \thirdnote{} till \eighthnote{} are available for further notes

% Simple summary

% Abstract (Do not insert blank lines, i.e. \\)
\abstract{\textbf{TODO} {[}Usually, 150-200 words overview of the
research paper you have conducted -- includes short introduction to the
research question, review to methods, review of key results, conclusion
for the results in line with the research paper.{]}}

% Keywords
\keyword{ADHD; behavioral activity; machine learning; time series}

% The fields PACS, MSC, and JEL may be left empty or commented out if not applicable
%\PACS{J0101}
%\MSC{}
%\JEL{}

%%%%%%%%%%%%%%%%%%%%%%%%%%%%%%%%%%%%%%%%%%
% Only for the journal Diversity
%\LSID{\url{http://}}

%%%%%%%%%%%%%%%%%%%%%%%%%%%%%%%%%%%%%%%%%%
% Only for the journal Applied Sciences:
%\featuredapplication{Authors are encouraged to provide a concise description of the specific application or a potential application of the work. This section is not mandatory.}
%%%%%%%%%%%%%%%%%%%%%%%%%%%%%%%%%%%%%%%%%%

%%%%%%%%%%%%%%%%%%%%%%%%%%%%%%%%%%%%%%%%%%
% Only for the journal Data:
%\dataset{DOI number or link to the deposited data set in cases where the data set is published or set to be published separately. If the data set is submitted and will be published as a supplement to this paper in the journal Data, this field will be filled by the editors of the journal. In this case, please make sure to submit the data set as a supplement when entering your manuscript into our manuscript editorial system.}

%\datasetlicense{license under which the data set is made available (CC0, CC-BY, CC-BY-SA, CC-BY-NC, etc.)}

%%%%%%%%%%%%%%%%%%%%%%%%%%%%%%%%%%%%%%%%%%
% Only for the journal Toxins
%\keycontribution{The breakthroughs or highlights of the manuscript. Authors can write one or two sentences to describe the most important part of the paper.}

%\setcounter{secnumdepth}{4}
%%%%%%%%%%%%%%%%%%%%%%%%%%%%%%%%%%%%%%%%%%


% tightlist command for lists without linebreak
\providecommand{\tightlist}{%
  \setlength{\itemsep}{0pt}\setlength{\parskip}{0pt}}




\begin{document}


%%%%%%%%%%%%%%%%%%%%%%%%%%%%%%%%%%%%%%%%%%

\hypertarget{introduction}{%
\section{Introduction}\label{introduction}}

Attention-deficit/hyperactivity disorder (ADHD) is a complex
neurodevelopmental disorder that shares symptoms with many other
conditions, and it is often misdiagnosed. It is estimated that 8.4\% of
children and 2.5\% of adults have ADHD, and presentation and assessment
are different in the two groups (\citet{noauthor_what_nodate}). There
are three main types of ADHD: inattentive presentation,
hyperactive/impulsive presentation, and combined presentation
(\citet{noauthor_what_nodate}). The inattentive type is characterized by
difficulty staying on task, sustaining focus, and staying organized
(\citet{noauthor_attention-deficithyperactivity_nodate}). Hyperactivity
is excessive movement and may present as restlessness or talking too
much in adults (\citet{noauthor_attention-deficithyperactivity_nodate}).
Impulsivity is when a person acts without thinking and may manifest as
desire for immediate rewards or the inability to delay gratification
(\citet{noauthor_attention-deficithyperactivity_nodate}). The combined
type is when both symptoms of the inattentive type and the
hyperactive/impulsive type are present
(\citet{noauthor_attention-deficithyperactivity_nodate}). ADHD can
impact individuals in many areas of their life such as
academic/professional, interpersonal relationships, and daily
functioning (\citet{noauthor_what_nodate}). In adults it can have far
reaching detrimental effects and lead to poor self-worth, sensitivity
towards criticism, and increased self-criticism
(\citet{noauthor_what_nodate}). However, sometimes ADHD is not
identified until a person is an adult if the symptoms were not
recognized, they had mild ADHD, or they managed sufficiently well until
demands of college/work
(\citet{noauthor_attention-deficithyperactivity_nodate2}). Due to the
harmful consequences ADHD can lead to, it is important that it is
diagnosed and treated.

There are many challenges to diagnosing ADHD, particularly in adults.
Adult ADHD symptoms are sometimes harder to discern than ADHD symptoms
in children (\citet{noauthor_adult_nodate}). Combining this with the
fact that adult ADHD symptoms are similar to those in other conditions
can make diagnosis difficult (\citet{noauthor_adult_nodate}). Stress,
illness, and other mental conditions such as anxiety or mood disorders
can all have symptoms that are similar to ADHD
(\citet{noauthor_adult_nodate},
\citet{noauthor_attention-deficithyperactivity_nodate2}. For example,
emotional dysregulation present in ADHD can be diagnosed with a mood
disorder or ADHD symptoms can be covered up by substance abuse
(\citet{pmid28830387}). Physicians are also usually more familiar with
mood and anxiety disorders, leading to misdiagnosis and delays in ADHD
treatment (\citet{pmid28830387}). Additionally, other mental health
conditions such as anxiety, mood, and substance use disorders are common
in adults with ADHD
(\citet{noauthor_attention-deficithyperactivity_nodate2}). Studies have
shown that 18.6\% to 53.3\% of people with ADHD have depression and
almost 50\% of people with ADHD have an anxiety disorder
(\citet{pmid28830387}). Some researchers suggest that in some cases
stress, depression, and anxiety may be manifesting due to undiagnosed or
untreated ADHD (\citet{pmid28830387}). These factors make ADHD difficult
to recognize and treat, leading to an under-diagnosis and
under-treatment of adult ADHD (\citet{pmid28830387}). Due to the
extensive effects ADHD can have, it is important that it is properly
diagnosed and treated.

There is no specific procedure to diagnose ADHD and psychiatrists,
neurologists, primary care doctors, clinical psychologists, or clinical
social workers can all diagnose adults with ADHD
(\citet{contributors_diagnosing_nodate}). Steps to getting a diagnosis
may include a physician using behavioral questionnaires to ask about the
impacts ADHD has, possible symptoms present in childhood, talking to a
parent or partner, and psychological tests
(\citet{contributors_diagnosing_nodate}). They may also test for
learning disabilities, other mental health conditions, or physical
illnesses to rule these options out
(\citet{contributors_diagnosing_nodate}).

A large part of the ADHD diagnosis process includes objective
information: the patient's perspective of themselves in behavioral
questionnaires, the perspective of parents and significant others, and
the opinion/view of the clinician. Rating scales, which are often used
in diagnosis, are systematic but they are not objective
(\citet{gualtieri_adhd_2005}). Raters are prone to let their view of the
subject and the outcome they want skew their results and different
raters often differ in their view of the same subject
(\citet{gualtieri_adhd_2005}). Patients are also evaluated by a
clinician during diagnosis, which can be a primary care physician in the
United States (\citet{contributors_diagnosing_nodate}). Recent research
has indicated in the United States children who are older for their
grade level are less likely to be diagnosed with ADHD
(\citet{dalsgaard_relative_2012}). However in Denmark only specialists
diagnose ADHD and these results were not replicated
(\citet{dalsgaard_relative_2012}). This supports the hypothesis that
non-specialists diagnosing ADHD could be the reason for the lower rate
of ADHD diagnosis in children who are older for their grade
(\citet{dalsgaard_relative_2012}). Clinicians may also be subject to
bias that affects their diagnoses of ADHD and there is evidence for
racial and gender disparities in diagnosis (\citet{noauthor_what_2022}).
Boys are more likely to be diagnosed with ADHD in childhood than girls,
though this may be due to different presentation of symptoms or
differences in building compensation skills
(\citet{noauthor_what_2022}). Black, Hispanic, and Asian children and
adults received ADHD diagnoses less often then their non-Hispanic white
counterparts (\citet{noauthor_what_2022}). In addition to rating scales
and clinical evaluations, computerized tests can also be used to help
diagnose ADHD. A computer test alone is not enough for a diagnosis, but
can supplement other diagnostic tools (\citet{gualtieri_adhd_2005}).
Continuous performance tests are used in ADHD diagnoses and test
vigilance or sustained attention (\citet{gualtieri_adhd_2005}). However,
there is limited correlation between CPT results and rating scales
(\citet{gualtieri_adhd_2005}). Additionally, the most common CPTs have
about a 85\% success rate in indicating ADHD in children that have been
diagnosed with ADHD and a false positive rate of 30\% in controls
(\citet{gualtieri_adhd_2005}). All together these make for a subjective,
potentially inaccurate diagnostic method.

Given the difficulties in diagnosing ADHD and the subjective methods
used, it's important that researchers investigate more accurate and
objective measures of diagnosis. \textbf{TODO: add previous research on
other predictive algorithms}

I will be continuing the research on predicting ADHD using machine
learning methods. \textbf{TODO: add why im using activity data}

\hypertarget{methods}{%
\section{Methods}\label{methods}}

\hypertarget{dataset}{%
\subsection{Dataset}\label{dataset}}

\textbf{TODO: update, add more description}

I am planning to use a dataset called HYPERAKTIV, which has 103
subjects. It contains subjects' sex, age, prescribed medications, which
psychiatric disorders are diagnosed (including ADHD/ADD), and if
substance abuse exists. For every subject it contains time series of
motor activity and heart rate over a 24-hour period and the results of a
neuropsychological computer test. Additionally, a paper was written on
it that contains robust metadata and the researchers did some beginning
exploratory work on the data using machine learning algorithms to
predict ADHD. I intend to use the time series of motor activity and age,
sex, psychiatric disorders and prescriptions medications to predict
ADHD.

I am planning to use a dataset called HYPERAKTIV, which has 103
subjects. It contains subjects' sex, age, prescribed medications, which
psychiatric disorders are diagnosed (including ADHD/ADD), and if
substance abuse exists. For every subject it contains time series of
motor activity and heart rate over a 24-hour period and the results of a
neuropsychological computer test. Additionally, a paper was written on
it that contains robust metadata and the researchers did some beginning
exploratory work on the data using machine learning algorithms to
predict ADHD. I intend to use the time series of motor activity and age,
sex, psychiatric disorders and prescriptions medications to predict
ADHD.

{[}Contrast this to what would have been your ideal dataset. How did you
arrive at choosing this dataset. Provide details on the dataset, size,
gender split and count, age, how it was collected, ethical concerns,
etc. Also provide details on data collection and access. Usually 100-200
words.{]}

\hypertarget{data-cleaning}{%
\subsection{Data Cleaning}\label{data-cleaning}}

In the Hyperaktiv data set the time series data for each subject is
stored in a separate CSV file. To analyze the data, I combined all of
the activity CSVs into one file, keeping track of which subject each
data point came from. \emph{TODO: add more detail if needed} There were
108 subjects, yet only some had activity data. I only used subjects with
activity data, which reduced the data set to 83.

\textbf{TODO: comment on if I make all activity csvs same length,
cleaning time series (seasonality, get rid of noise, etc.)}

\textbf{TODO: Add about heart rate if you get there}

\hypertarget{variables-of-interest}{%
\subsection{Variables of Interest}\label{variables-of-interest}}

\textbf{TODO, also include summary stats of variables}

{[}Describe the variables that you are using in your datasets. Describe
the rationale for using these variables. Varies in length.{]}

\hypertarget{other-important-features}{%
\subsection{Other Important Features}\label{other-important-features}}

\textbf{TODO: Researchers did some preliminary ML on dataset - discuss}

{[}Also comment on other things which are relevant to your research
paper. Varies in length.{]}

\hypertarget{data-analyses}{%
\subsection{Data Analyses}\label{data-analyses}}

\textbf{TODO: this is info from draft, update with actual procedure
taken}

First, I will calculate summary statistics of the variables I'm using
introductory data visualizations. It looks like to be compared, time
series have to be the same length. Since the data I am using is not all
the same length, if I am unable to find a work-around I will have to
preprocess the data by finding the mean of the lengths and then
truncating the series that are too long and padding the ones that are
too short.

\begin{verbatim}
My research question involves investigating if ADHD can be predicted using motor activity. I would like to try and use different time series classification algorithms to see if this can be predicted. I will start by using ROCKET and HIVE-COTE, which are ensemble classifiers based on the time series classification algorithms mentioned next. I will then try distance-based (KNN with dynamic time warping), interval-based (TimeSeriesForest), dictionary-based (BOSS, cBOSS), Frequency-based (RISE — like TimeSeriesForest but with other features), and shapelet-based (Shapelet Transform Classifier) algorithms and compare the results (accuracy, F1, sensitivity). 
\end{verbatim}

Before modeling of performing significance tests on the data, I will
check if the data is stationary by using a Phillips-Perron Unit Root
Test or by plotting the sample autocorrelation function. If it isn't
stationary I will remove the trends or seasonality from the data
(trends: difference the data or perform least squares trend removal,
seasonality: seasonal differencing, seasonal means, moving averages). To
compare the two groups and see if there is a significant difference
between them, I am going to try fitting a model to each group of the
time series, use the model to simulate new data, and then run statistics
on the simulated data. I am going to continue looking to see if there is
a more direct method for comparing two groups of time series data.

In addition to looking at the data as time series data, I could also try
picking out features of the motor activity data and see if there is a
significant difference between ADHD vs non-ADHD groups and if prediction
is possible. I could pick out features such as averages, standard
deviations, or number of peaks in the motor activity time series data.
This would simplify the modeling, significance testing, and prediction
algorithms steps since this data would be easier to use than time series
data. It would also allow me to include confounding variables such as
sex, age, and other psychiatric disorders because I have not yet found a
way to include confounding variables in time series data. If I have time
I could also include heart rate of people as a predictor variable in
addition to motor activity to predict if people have ADHD or not. There
has been some evidence of difference in heart rates between the two
groups, so it could be beneficial to look at this (Hicks et al., 2021)

{[}What analyses are you doing why. Varies in length.{]}

\hypertarget{results}{%
\section{Results}\label{results}}

{[}Summary of results analyses 1{]}

{[}Add visualizations from analyses. Varies in length.{]}

{[}Summary of results analyses 2{]}

{[}Add visualizations from analyses. Varies in length.{]}

\hypertarget{discussion}{%
\section{Discussion}\label{discussion}}

\begin{itemize}
\tightlist
\item
  Talk about what the results mean
\item
  Depending on results - could have positive impact on diagnosing ADHD
\item
  Diagnosis in data is still coming from people - so still has
  subjectivity
\item
  Couldn't include confounding variables??
\item
  Most practitioners don't have the technical skill to do it themselves
\end{itemize}

{[}Explain what your results mean in the context of the literature cited
in the introduction. Minimum of 750 words.{]}

\hypertarget{future-directions}{%
\section{Future Directions}\label{future-directions}}

\begin{itemize}
\tightlist
\item
  Larger sample size
\item
  Combine with heart rate data
\item
  Could try on children
\item
  Make accessible to practitioners if it works
\item
  If I found another dataset, I could also explore if lifestyle factors
  (maybe things such as amount of exercise, level of satisfaction,
  income level, etc.) can be used to predict ADHD.
\end{itemize}

\textbf{TODO: change citation style}

{[}How could someone continue the work? Around 150-200 words.{]}

\hypertarget{personal-reflection}{%
\section{Personal Reflection}\label{personal-reflection}}

{[}What was your original research question? Why did you have to
deviate? What ideal dataset would be recommend collecting? How was this
process of developing a research paper? 250 words{]}

\hypertarget{code-availability}{%
\section{Code availability}\label{code-availability}}

All analysis code for this article is available at:
\url{https://github.com/i-m-foster/sds300np-ireneFoster}

\hypertarget{acknowledgements}{%
\section{Acknowledgements}\label{acknowledgements}}

{[}Mention who you would like to thank. Any grants or people. Varies in
length.{]}

% %%%%%%%%%%%%%%%%%%%%%%%%%%%%%%%%%%%%%%%%%%
% %% optional
% \supplementary{The following are available online at www.mdpi.com/link, Figure S1: title, Table S1: title, Video S1: title.}
%
% % Only for the journal Methods and Protocols:
% % If you wish to submit a video article, please do so with any other supplementary material.
% % \supplementary{The following are available at www.mdpi.com/link: Figure S1: title, Table S1: title, Video S1: title. A supporting video article is available at doi: link.}

\vspace{6pt}

%%%%%%%%%%%%%%%%%%%%%%%%%%%%%%%%%%%%%%%%%%

%%%%%%%%%%%%%%%%%%%%%%%%%%%%%%%%%%%%%%%%%%

%%%%%%%%%%%%%%%%%%%%%%%%%%%%%%%%%%%%%%%%%%
\conflictsofinterest{The authors declare no conflict of interest.}

%%%%%%%%%%%%%%%%%%%%%%%%%%%%%%%%%%%%%%%%%%
%% optional


%%%%%%%%%%%%%%%%%%%%%%%%%%%%%%%%%%%%%%%%%%
% Citations and References in Supplementary files are permitted provided that they also appear in the reference list here.

%=====================================
% References, variant A: internal bibliography
%=====================================
%\reftitle{References}
%\begin{thebibliography}{999}
% Reference 1
%\bibitem[Author1(year)]{ref-journal}
%Author1, T. The title of the cited article. {\em Journal Abbreviation} {\bf 2008}, {\em 10}, 142--149.
% Reference 2
%\bibitem[Author2(year)]{ref-book}
%Author2, L. The title of the cited contribution. In {\em The Book Title}; Editor1, F., Editor2, A., Eds.; Publishing House: City, Country, 2007; pp. 32--58.
%\end{thebibliography}

% The following MDPI journals use author-date citation: Arts, Econometrics, Economies, Genealogy, Humanities, IJFS, JRFM, Laws, Religions, Risks, Social Sciences. For those journals, please follow the formatting guidelines on http://www.mdpi.com/authors/references
% To cite two works by the same author: \citeauthor{ref-journal-1a} (\citeyear{ref-journal-1a}, \citeyear{ref-journal-1b}). This produces: Whittaker (1967, 1975)
% To cite two works by the same author with specific pages: \citeauthor{ref-journal-3a} (\citeyear{ref-journal-3a}, p. 328; \citeyear{ref-journal-3b}, p.475). This produces: Wong (1999, p. 328; 2000, p. 475)

%=====================================
% References, variant B: external bibliography
%=====================================
\reftitle{References}
\externalbibliography{yes}
\bibliography{mybibfile.bib}

%%%%%%%%%%%%%%%%%%%%%%%%%%%%%%%%%%%%%%%%%%
%% optional

%% for journal Sci
%\reviewreports{\\
%Reviewer 1 comments and authors’ response\\
%Reviewer 2 comments and authors’ response\\
%Reviewer 3 comments and authors’ response
%}

%%%%%%%%%%%%%%%%%%%%%%%%%%%%%%%%%%%%%%%%%%


\end{document}
