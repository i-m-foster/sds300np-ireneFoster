%  LaTeX support: latex@mdpi.com
%  In case you need support, please attach all files that are necessary for compiling as well as the log file, and specify the details of your LaTeX setup (which operating system and LaTeX version / tools you are using).

%=================================================================
\documentclass[,article,,moreauthors,pdftex]{mdpi}

% If you would like to post an early version of this manuscript as a preprint, you may use preprint as the journal and change 'submit' to 'accept'. The document class line would be, e.g., \documentclass[preprints,article,accept,moreauthors,pdftex]{mdpi}. This is especially recommended for submission to arXiv, where line numbers should be removed before posting. For preprints.org, the editorial staff will make this change immediately prior to posting.

%% Some pieces required from the pandoc template
\setlist[itemize]{leftmargin=*,labelsep=5.8mm}
\setlist[enumerate]{leftmargin=*,labelsep=4.9mm}


%--------------------
% Class Options:
%--------------------
%----------
% journal
%----------
% Choose between the following MDPI journals:
% acoustics, actuators, addictions, admsci, aerospace, agriculture, agriengineering, agronomy, algorithms, animals, antibiotics, antibodies, antioxidants, applsci, arts, asc, asi, atmosphere, atoms, axioms, batteries, bdcc, behavsci , beverages, bioengineering, biology, biomedicines, biomimetics, biomolecules, biosensors, brainsci , buildings, cancers, carbon , catalysts, cells, ceramics, challenges, chemengineering, chemistry, chemosensors, children, cleantechnol, climate, clockssleep, cmd, coatings, colloids, computation, computers, condensedmatter, cosmetics, cryptography, crystals, dairy, data, dentistry, designs , diagnostics, diseases, diversity, drones, econometrics, economies, education, electrochem, electronics, energies, entropy, environments, epigenomes, est, fermentation, fibers, fire, fishes, fluids, foods, forecasting, forests, fractalfract, futureinternet, futurephys, galaxies, games, gastrointestdisord, gels, genealogy, genes, geohazards, geosciences, geriatrics, hazardousmatters, healthcare, heritage, highthroughput, horticulturae, humanities, hydrology, ijerph, ijfs, ijgi, ijms, ijns, ijtpp, informatics, information, infrastructures, inorganics, insects, instruments, inventions, iot, j, jcdd, jcm, jcp, jcs, jdb, jfb, jfmk, jimaging, jintelligence, jlpea, jmmp, jmse, jnt, jof, joitmc, jpm, jrfm, jsan, land, languages, laws, life, literature, logistics, lubricants, machines, magnetochemistry, make, marinedrugs, materials, mathematics, mca, medicina, medicines, medsci, membranes, metabolites, metals, microarrays, micromachines, microorganisms, minerals, modelling, molbank, molecules, mps, mti, nanomaterials, ncrna, neuroglia, nitrogen, notspecified, nutrients, ohbm, particles, pathogens, pharmaceuticals, pharmaceutics, pharmacy, philosophies, photonics, physics, plants, plasma, polymers, polysaccharides, preprints , proceedings, processes, proteomes, psych, publications, quantumrep, quaternary, qubs, reactions, recycling, religions, remotesensing, reports, resources, risks, robotics, safety, sci, scipharm, sensors, separations, sexes, signals, sinusitis, smartcities, sna, societies, socsci, soilsystems, sports, standards, stats, surfaces, surgeries, sustainability, symmetry, systems, technologies, test, toxics, toxins, tropicalmed, universe, urbansci, vaccines, vehicles, vetsci, vibration, viruses, vision, water, wem, wevj

%---------
% article
%---------
% The default type of manuscript is "article", but can be replaced by:
% abstract, addendum, article, benchmark, book, bookreview, briefreport, casereport, changes, comment, commentary, communication, conceptpaper, conferenceproceedings, correction, conferencereport, expressionofconcern, extendedabstract, meetingreport, creative, datadescriptor, discussion, editorial, essay, erratum, hypothesis, interestingimages, letter, meetingreport, newbookreceived, obituary, opinion, projectreport, reply, retraction, review, perspective, protocol, shortnote, supfile, technicalnote, viewpoint
% supfile = supplementary materials

%----------
% submit
%----------
% The class option "submit" will be changed to "accept" by the Editorial Office when the paper is accepted. This will only make changes to the frontpage (e.g., the logo of the journal will get visible), the headings, and the copyright information. Also, line numbering will be removed. Journal info and pagination for accepted papers will also be assigned by the Editorial Office.

%------------------
% moreauthors
%------------------
% If there is only one author the class option oneauthor should be used. Otherwise use the class option moreauthors.

%---------
% pdftex
%---------
% The option pdftex is for use with pdfLaTeX. If eps figures are used, remove the option pdftex and use LaTeX and dvi2pdf.

%=================================================================
\firstpage{1}
\makeatletter
\setcounter{page}{\@firstpage}
\makeatother
\pubvolume{xx}
\issuenum{1}
\articlenumber{5}
\pubyear{2019}
\copyrightyear{2019}
%\externaleditor{Academic Editor: name}
\history{Received: date; Accepted: date; Published: date}
\updates{yes} % If there is an update available, un-comment this line

%% MDPI internal command: uncomment if new journal that already uses continuous page numbers
%\continuouspages{yes}

%------------------------------------------------------------------
% The following line should be uncommented if the LaTeX file is uploaded to arXiv.org
%\pdfoutput=1

%=================================================================
% Add packages and commands here. The following packages are loaded in our class file: fontenc, calc, indentfirst, fancyhdr, graphicx, lastpage, ifthen, lineno, float, amsmath, setspace, enumitem, mathpazo, booktabs, titlesec, etoolbox, amsthm, hyphenat, natbib, hyperref, footmisc, geometry, caption, url, mdframed, tabto, soul, multirow, microtype, tikz

%=================================================================
%% Please use the following mathematics environments: Theorem, Lemma, Corollary, Proposition, Characterization, Property, Problem, Example, ExamplesandDefinitions, Hypothesis, Remark, Definition
%% For proofs, please use the proof environment (the amsthm package is loaded by the MDPI class).

%=================================================================
% Full title of the paper (Capitalized)
\Title{Classifying ADHD Using Activity Time Series Data}

% Authors, for the paper (add full first names)
\Author{Irene
Foster$^{1,*}$\href{https://orcid.org/0000-0001-9681-4786}{\orcidicon}}

% Authors, for metadata in PDF
\AuthorNames{Irene Foster}

% Affiliations / Addresses (Add [1] after \address if there is only one affiliation.)
\address{%
$^{1}$ \quad Smith College - Department of Statistical \& Data Sciences
Northampton, MA, USA; \\
}
% Contact information of the corresponding author
\corres{Correspondence: \href{mailto:ifoster25@smith.edu}{\nolinkurl{ifoster25@smith.edu}}}

% Current address and/or shared authorship








% The commands \thirdnote{} till \eighthnote{} are available for further notes

% Simple summary

% Abstract (Do not insert blank lines, i.e. \\)
\abstract{ADHD is a neurodevelopmental disorder which can have wide
reaching detrimental impacts on a person's life. Due to having similar
symptoms as other psychiatric disorders and often being comorbid with
other psychiatric disorders ADHD is often misdiagnosed. One common
diagnostic tool is behavioral questionnaires from the patient, parents,
and the opinion of the clinician, which are all subjective. This paper
explores using time series machine learning algorithms to classify
whether people have ADHD or not to investigate a more objective ADHD
diagnostic tool. The data used is the HYPERAKTIV motor activity dataset.
After removing noise and trends from the time series motor activity
data, eight classification algorithms from the sktime package were
applied to the data. Four of the algorithms resulted in all positive
classifications. The remaining four algorithms also provided low
accuracy, precision, recall, and F1-scores. The highest scoring
algorithm that did not provide all positive classifications has an
accuracy of 0.47, precision 0f 0.55, recall of 0.50, and F1-score of
0.52. Time series classification algorithms appear to be a poor
diagnostic tool for ADHD, and are outperformed by previous research.}

% Keywords
\keyword{ADHD; behavioral activity; machine learning; time series}

% The fields PACS, MSC, and JEL may be left empty or commented out if not applicable
%\PACS{J0101}
%\MSC{}
%\JEL{}

%%%%%%%%%%%%%%%%%%%%%%%%%%%%%%%%%%%%%%%%%%
% Only for the journal Diversity
%\LSID{\url{http://}}

%%%%%%%%%%%%%%%%%%%%%%%%%%%%%%%%%%%%%%%%%%
% Only for the journal Applied Sciences:
%\featuredapplication{Authors are encouraged to provide a concise description of the specific application or a potential application of the work. This section is not mandatory.}
%%%%%%%%%%%%%%%%%%%%%%%%%%%%%%%%%%%%%%%%%%

%%%%%%%%%%%%%%%%%%%%%%%%%%%%%%%%%%%%%%%%%%
% Only for the journal Data:
%\dataset{DOI number or link to the deposited data set in cases where the data set is published or set to be published separately. If the data set is submitted and will be published as a supplement to this paper in the journal Data, this field will be filled by the editors of the journal. In this case, please make sure to submit the data set as a supplement when entering your manuscript into our manuscript editorial system.}

%\datasetlicense{license under which the data set is made available (CC0, CC-BY, CC-BY-SA, CC-BY-NC, etc.)}

%%%%%%%%%%%%%%%%%%%%%%%%%%%%%%%%%%%%%%%%%%
% Only for the journal Toxins
%\keycontribution{The breakthroughs or highlights of the manuscript. Authors can write one or two sentences to describe the most important part of the paper.}

%\setcounter{secnumdepth}{4}
%%%%%%%%%%%%%%%%%%%%%%%%%%%%%%%%%%%%%%%%%%


% tightlist command for lists without linebreak
\providecommand{\tightlist}{%
  \setlength{\itemsep}{0pt}\setlength{\parskip}{0pt}}



\usepackage{booktabs}
\usepackage{longtable}
\usepackage{array}
\usepackage{multirow}
\usepackage{wrapfig}
\usepackage{float}
\usepackage{colortbl}
\usepackage{pdflscape}
\usepackage{tabu}
\usepackage{threeparttable}
\usepackage{threeparttablex}
\usepackage[normalem]{ulem}
\usepackage{makecell}
\usepackage{xcolor}

\begin{document}


%%%%%%%%%%%%%%%%%%%%%%%%%%%%%%%%%%%%%%%%%%

\hypertarget{introduction}{%
\section{Introduction}\label{introduction}}

Attention-deficit/hyperactivity disorder (ADHD) is a complex
neurodevelopmental disorder that shares symptoms with many other
conditions, and it is often misdiagnosed. It is estimated that 8.4\% of
children and 2.5\% of adults have ADHD, and presentation and assessment
are different in the two groups (\citet{noauthor_what_nodate}). There
are three main types of ADHD: inattentive presentation,
hyperactive/impulsive presentation, and combined presentation
(\citet{noauthor_what_nodate}). The inattentive type is characterized by
difficulty staying on task, sustaining focus, and staying organized
(\citet{noauthor_attention-deficithyperactivity_nodate}). Hyperactivity
is excessive movement and may present as restlessness or talking too
much in adults (\citet{noauthor_attention-deficithyperactivity_nodate}).
Impulsivity is when a person acts without thinking and may manifest as
desire for immediate rewards or the inability to delay gratification
(\citet{noauthor_attention-deficithyperactivity_nodate}). The combined
type is when both symptoms of the inattentive type and the
hyperactive/impulsive type are present
(\citet{noauthor_attention-deficithyperactivity_nodate}). ADHD can
impact individuals in many areas of their life such as
academic/professional, interpersonal relationships, and daily
functioning (\citet{noauthor_what_nodate}). In adults it can have far
reaching detrimental effects and lead to poor self-worth, sensitivity
towards criticism, and increased self-criticism
(\citet{noauthor_what_nodate}). However, sometimes ADHD is not
identified until a person is an adult if the symptoms were not
recognized, they had mild ADHD, or they managed sufficiently well until
demands of college/work
(\citet{noauthor_attention-deficithyperactivity_nodate2}). Due to the
harmful consequences ADHD can lead to, it is important that it is
diagnosed and treated.

There are many challenges to diagnosing ADHD, particularly in adults.
Adult ADHD symptoms are sometimes harder to discern than ADHD symptoms
in children (\citet{noauthor_adult_nodate}). Combining this with the
fact that adult ADHD symptoms are similar to those in other conditions
can make diagnosis difficult (\citet{noauthor_adult_nodate}). Stress,
illness, and other mental conditions such as anxiety or mood disorders
can all have symptoms that are similar to ADHD
(\citet{noauthor_adult_nodate},
\citet{noauthor_attention-deficithyperactivity_nodate2}). For example,
emotional dysregulation present in ADHD can be diagnosed with a mood
disorder or ADHD symptoms can be covered up by substance abuse
(\citet{pmid28830387}). Physicians are also usually more familiar with
mood and anxiety disorders, leading to misdiagnosis and delays in ADHD
treatment (\citet{pmid28830387}). Additionally, other mental health
conditions such as anxiety, mood, and substance use disorders are common
in adults with ADHD
(\citet{noauthor_attention-deficithyperactivity_nodate2}). Studies have
shown that 18.6\% to 53.3\% of people with ADHD have depression and
almost 50\% of people with ADHD have an anxiety disorder
(\citet{pmid28830387}). Some researchers suggest that in some cases
stress, depression, and anxiety may be manifesting due to undiagnosed or
untreated ADHD (\citet{pmid28830387}). These factors make ADHD difficult
to recognize and treat, leading to an under-diagnosis and
under-treatment of adult ADHD (\citet{pmid28830387}). Due to the
extensive effects ADHD can have, it is important that it is properly
diagnosed and treated.

The procedure to diagnose ADHD is not standardized. Psychiatrists,
neurologists, primary care doctors, clinical psychologists, or clinical
social workers can all diagnose adults with ADHD
(\citet{contributors_diagnosing_nodate}). Steps to getting a diagnosis
may include a physician using behavioral questionnaires to ask about the
impacts ADHD has, possible symptoms present in childhood, talking to a
parent or partner, and psychological tests
(\citet{contributors_diagnosing_nodate}). They may also test for
learning disabilities, other mental health conditions, or physical
illnesses to rule these options out
(\citet{contributors_diagnosing_nodate}).

A large part of the ADHD diagnosis process includes subjective
information: the patient's perspective of themselves in behavioral
questionnaires, the perspective of parents and significant others, and
the opinion/view of the clinician. Rating scales, which are often used
in diagnosis, are systematic but they are not objective
(\citet{gualtieri_adhd_2005}). Raters are prone to let their view of the
subject and the outcome they want skew their results and different
raters often differ in their view of the same subject
(\citet{gualtieri_adhd_2005}). Patients are also evaluated by a
clinician during diagnosis, which can be a primary care physician in the
United States (\citet{contributors_diagnosing_nodate}). Recent research
has indicated in the United States children who are older for their
grade level are less likely to be diagnosed with ADHD
(\citet{dalsgaard_relative_2012}). However in Denmark, where only
specialists diagnose ADHD, these results were not replicated
(\citet{dalsgaard_relative_2012}). This supports the hypothesis that
non-specialists diagnosing ADHD could be the reason for the lower rate
of ADHD diagnosis in children who are older for their grade
(\citet{dalsgaard_relative_2012}). Clinicians may also be subject to
bias that affects their diagnoses of ADHD and there is evidence for
racial and gender disparities in diagnosis (\citet{noauthor_what_2022}).
Boys are more likely to be diagnosed with ADHD in childhood than girls,
though this may be due to different presentation of symptoms or
differences in building compensation skills
(\citet{noauthor_what_2022}). Black, Hispanic, and Asian children and
adults received ADHD diagnoses less often then their non-Hispanic white
counterparts (\citet{noauthor_what_2022}). In addition to rating scales
and clinical evaluations, computerized tests can also be used to help
diagnose ADHD. A computer test alone is not enough for a diagnosis, but
can supplement other diagnostic tools (\citet{gualtieri_adhd_2005}).
Continuous performance tests (CPTs) are used in ADHD diagnoses and test
vigilance or sustained attention (\citet{gualtieri_adhd_2005}). However,
there is limited correlation between CPT results and rating scales
(\citet{gualtieri_adhd_2005}). Additionally, the most common CPTs have
about a 85\% success rate in indicating ADHD in children that have been
diagnosed with ADHD and a false positive rate of 30\% in controls
(\citet{gualtieri_adhd_2005}). All together these make for a subjective,
potentially inaccurate diagnostic method.

Given the difficulties in diagnosing ADHD and the subjective methods
used, it's important that researchers investigate more accurate and
objective measures of diagnosis. Previous research has already been
conducted on using machine learning methods to classify ADHD on a
variety of different data types. Researchers using Conners' Adult ADHD
Rating Scales (CAARS-S: S), which is a rating scale for ADHD symptoms
present, applied a LightGBM algorithm to differentiate between ADHD,
obesity, problematic gambling, and a control group
(\citet{christiansen_use_2020}). They were able to achieve a global
accuracy of 0.80 (\citet{christiansen_use_2020}). Another study applied
extreme learning machine (ELM) and SVM algorithms to structural MRI data
from the the ADHD-200 Global Competition, achieving a maximum accuracy
of 90.18\% using the ELM (\citet{peng_extreme_2013}). One study used
data from the 2018--2019 National Survey of Children's Health
(\citet{maniruzzaman_predicting_2022}). After performing feature
selection on the data, they applied eight classification algorithms,
finding a highest accuracy of 85.5\% using a random forest based
classifier (\citet{maniruzzaman_predicting_2022}). Real-world clinical
data featuring multiple psychiatric conditions has been used with a SVM
classifier to predict if patients have ADHD with an accuracy of 66.1\%
(\citet{mikolas_training_2022}). Motor activity, specifically the motor
activity dataset from HYPERAKTIV has also been used in previous studies
(\citet{10.1145/3458305.3478454}, \citet{kaur_accurate_2022}). The
studies used feature extraction and the other study used principal
component analysis before applying machine learning methods, resulting
in highest accuracies of 72\% and 98.5\% respectively
(\citet{10.1145/3458305.3478454}, \citet{kaur_accurate_2022}). These
studies suggest that applying machine learning to classify ADHD is
possible.

I will be continuing the research on predicting ADHD using machine
learning methods, focusing on time series classification algorithms
applied to motor activity data. Actigraph data has been recognized as a
potential tool for ADHD diagnoses (\citet{10.1145/3458305.3478454}). A
study on circadian rhythm also found increased restlessness in sleep at
the end of the night and increased activity in the afternoon for people
with ADHD versus people without ADHD (\citet{10.1145/3458305.3478454}).
A review of 24 studies on motor activity in children show that children
with ADHD have higher average activity during structured sessions,
similar sleep durations, and a moderately different sleep pattern
compared to children without ADHD (\citet{de_crescenzo_use_2016}). These
studies suggest that motor activity data has potential to be used as a
classifier for ADHD.

\hypertarget{methods}{%
\section{Methods}\label{methods}}

\hypertarget{dataset}{%
\subsection{Dataset}\label{dataset}}

The dataset used in this paper is the HYPERAKTIV dataset, which contains
time series activity and heart rate data for 85 subjects
(\citet{10.1145/3458305.3478454}). Motor activity data was collected
with Actiwatch (Cambridge Neurotechnology Ltd, England, model AW4) which
is a wrist-worn actigraph. It records intensity, amount, and duration of
movement in the x, y, and z-axes with a sampling frequency of 32 Hz.
Activity data for 45 subjects with ADHD was recorded for 6.6 ± 1.3 days
and 7.2 ± 0.9 days for the 40 controls. There are 103 subjects total,
but only 85 recorded activity data. Heart rate was collected using
Actiheart (Cambridge Neurotechnology Ltd, England), a chest-worn ECG
monitor. It contains the raw data without correction of the time between
beats in milliseconds. 80 out of the 103 subjects recorded heart rate
data. 38 subjects with ADHD recorded heart rate data for an average of
20.5 ± 3.9 hours and 42 controls recorded heart rate data for 21 ± 4.3
hours.

In addition to the heart rate and motor activity data, the dataset also
includes every patient's 360 CPT-II test results along with the omission
and commission errors, and the ADHD Confidence Index. There is also a
dataset containing background information on each subject such as age,
sex, type of medications prescribed, whether the patient was diagnosed
with ADHD, if the ADD subtype was present, and other diagnosed
psychiatric disorders. The psychiatric disorders were diagnosed by two
experienced and certified psychiatrists for all patients using the
Mini-International Neuropsychiatric Interview. Psychiatric disorder
variables are: ADHD, ADD, BIPOLAR, UNIPOLAR, ANXIETY, potential
substance abuse problems, and OTHER. Of the 85 people who recorded
activity data there were 44 males and 41 females, 23 people 17-29 years
old, 26 patients who were 30-39 years old, 24 who were 40-49 years old,
and 14 who were 50-67 years old. Of the subjects who were diagnosed with
ADHD 73\% were not on medications and only one person was on stimulants
(See Table 1 for more dataset variable breakdown).

\begin{table}

\caption{\label{tab:tab:CellCount Plugin Summary}HYPERAKTIV Dataset Demographic Breakdown}
\centering
\begin{tabular}[t]{c|c|c}
\hline
Variable & ADHD & Control\\
\hline
Count & 45 & 40\\
\hline
Sex(m/f) & 24/21 & 20/20\\
\hline
Age(1/2/3/4) & 11/14/14/6 & 12/12/10/6\\
\hline
ADD & 23 & 0\\
\hline
UNIPOLAR & 16 & 10\\
\hline
BIPOLAR & 16 & 20\\
\hline
ANXIETY & 18 & 26\\
\hline
SUBSTANCE & 12 & 7\\
\hline
OTHER & 11 & 16\\
\hline
\multicolumn{3}{l}{\rule{0pt}{1em}\textit{Note: }}\\
\multicolumn{3}{l}{\rule{0pt}{1em}Ages were split into four categories: 1 = 17-29 years, 2 = 30-39 years, 3 = 40-49 years, and 4 = 50-67 years}\\
\end{tabular}
\end{table}

\hypertarget{data-cleaning}{%
\subsection{Data Cleaning}\label{data-cleaning}}

In the Hyperaktiv data set the time series data for each subject is
stored in a separate CSV file. To analyze the data, all of the
individual activity .CSVs were combined into one data frame, keeping
track of which subject each data point came from. There were 103
subjects, but only 83 subjects had activity data, which reduced the data
set. The motor activity data was recorded for multiple days for each
participant (6.6 ± 1.3 days for people with ADH and 7.2 ± 0.9 days for
the controls). For every subject, the multiple day data was averaged so
each subject had one 24 hour period of activity data with the an average
activity data measurement for every hour. For time series classification
models in the sktime package (\citet{loning_sktimesktime_2022},
\citet{loning_sktime_nodate}) the time series all need to be the same
length. However, the activity for each subject is different in the
HYPERAKTIV package since it is based on how active they were. Averaging
the data into one 24 hour period also creates time series that are all
the same length. The overall average activity data between people
diagnosed with ADHD and people not diagnosed with ADHD is very similar
and contains a lot of noise (see Fig. 1). To fix this, for each subject
a rolling mean was applied to the data to get rid of the noise, the data
points were differenced (each data point had the value before it
subtracted from it), and then another rolling mean was applied to get
rid of more noise. The overall averaged activity data between the groups
is still similar (see Fig. 1).

\begin{figure}
\includegraphics[width=0.48\linewidth]{FinalReport_files/figure-latex/unnamed-chunk-2-1} \includegraphics[width=0.48\linewidth]{FinalReport_files/figure-latex/unnamed-chunk-2-2} \caption{Average activity data for control and ADHD groups (left) and average activity data for control and ADHD groups after removing noise and trends in data (right)}\label{fig:unnamed-chunk-2}
\end{figure}

\hypertarget{variables-of-interest}{%
\subsection{Variables of Interest}\label{variables-of-interest}}

This paper focuses on the time series activity data from the HYPERAKTIV
data set. It contains data from a wrist-worn actipgrah, which measures
gross motor activity, for 85 subjects for approximately one week of
measurements (\citet{10.1145/3458305.3478454}). The length of each
subject's activity data is different, depending on how active they were.
The average number of activity data points was 9977.5 (sd = 1669.5, min
= 3166, max = 14607). Most of the subjects had data around the same
length as the average (see Fig. 1). In the HYPERAKTIV datset an
actigrpah with a sampling frequency of 32 Hz was used and many movements
over 0.05g were recorded (\citet{10.1145/3458305.3478454}). See Figure.
1 for the average motor activity for the ADHD and control groups over a
24 hour period. Confounding variables from the HYPERAKTIV demographic
dataset such as sex, age, and other pyshiatric disorders were not
included due to inability to include confounding variables in time
series classification methods.

\begin{figure}

{\centering \includegraphics[width=0.5\linewidth]{FinalReport_files/figure-latex/unnamed-chunk-3-1} 

}

\caption{Frequency of Length of Activity Data}\label{fig:unnamed-chunk-3}
\end{figure}

\hypertarget{other-important-features}{%
\subsection{Other Important Features}\label{other-important-features}}

The HYPERAKTIV dataset has already been used to classify if people were
diagnosed with ADHD based on motor activity data. The researchers who
published HYPERAKTIV (\citet{10.1145/3458305.3478454}) did some
exploratory analysis on the data. They used the tsfresh package to
extract and select features from the motor activity data, and then
applied machine learning algorithms and simple prediction algorithms to
predict ADHD. The accuracies ranged from 58\% to 72\%. A subsequent
paper applied principal component analysis (PCA) to the motor activity
data from the HYPERAKTIV data set and then used six machine learning
classification algorithms (\citet{kaur_accurate_2022}). They were able
to achieve accuracies from 80.39\% to 98.43\%. Five out of the six
algorithms used have a higher accuracy than the most common CPTs, which
is about 85\% (\citet{gualtieri_adhd_2005}). The best method from this
paper for classifying ADHD based on motor activity was a principle
component analysis followed by classification using a SVM machine
learning algorithm, which achieved an accuracy of 98.43\%, F-measure of
98.42\%, recall of 98.33\%, and precision of 98.56\%
(\citet{kaur_accurate_2022}).

\hypertarget{data-analyses}{%
\subsection{Data Analyses}\label{data-analyses}}

To investigate the research question if people with ADHD can be
classified using motor activity, time series classification algorithms
were applied to the HYPERAKTIV motor activity data after it had been
processed to remove the trends and noise. Eight different classification
algorithms from the sktime package (\citet{loning_sktime_nodate},
\citet{loning_sktimesktime_2022}) were applied including kernel based,
dictionary based, distance based, and deep learning types. The data was
split into training and testing groups (80\% and 20\% respectively) and
the same training and testing splits were used for all algorithms. The
algorithms used for analysis were IndividualTDE a temporal dictionary
ensemble classifier, KNeighborsTimeSeriesClassifier an adapted
KNeighborsClassifier for time series data, DummyClassifier a classifier
that ignores input features, and TimeSeriesSVC a Support Vector
Classifier for time series data. Other algorithms used were ROCKET which
transforms time series using random convolutional kernels and then
classifies it using RidgeClassifierCV, Arsenal which is an ensemble of
ROCKET transformers, FCNClassifier a fully connected neural network
classifier, and LSTMFCNClassifier a Long Short Term Memory Fully
Convolutional Network classifier for time series data. All algorithms
were used with the default parameters. These algorithms were selected
because they work on the multivariate HYPERAKTIV activity data and
represent a range of different types of time series classification
algorithms.

After using the sktime time series classification algorithms for the
predictions, Scikit-learn (\citet{scikit-learn}) was then used to obtain
the accuracy, precision, recall, F1-score, and Matthews correlation
coefficient for each algorithm, as recommended by the authors of the
dataset (\citet{10.1145/3458305.3478454}). Accuracy is how many
predictions the algorithm got right, precision measures how many
positives were correct, and recall measures whether all the positive
predictions were correctly identified. The F1-score measures a test's
accuracy, taking into account the precision and recall
(\citet{korstanje_f1_2021}). Matthews correlation coefficient measures
the difference between the predicted values and the actual values
(\citet{noauthor_matthewss_nodate}).

\hypertarget{results}{%
\section{Results}\label{results}}

\begin{table}

\caption{\label{tab:tab:CellCount Plugin Summary}sktime Time Series Classification Algorithm Results}
\centering
\begin{tabular}[t]{c|c|c|c|c|c}
\hline
Algorithm & Accuracy & Precision & Recall & F1Score & Matthews\\
\hline
ROCKET & 0.59 & 0.59 & 1.0 & 0.74 & 0,0\\
\hline
IndividualTDE & 0.18 & 0.17 & 0.1 & 0.13 & -0.63\\
\hline
KNeighborsTimeSeriesClassifier & 0.47 & 0.55 & 0.5 & 0.52 & -0.07\\
\hline
DummyClassifier & 0.59 & 0.59 & 1.0 & 0.74 & 0.0\\
\hline
TimeSeriesSVC & 0.59 & 0.59 & 1.0 & 0.74 & 0.0\\
\hline
Arsenal & 0.59 & 0.59 & 1.0 & 0.74 & 0.0\\
\hline
FCNClassifier & 0.35 & 0.45 & 0.5 & 0.48 & -0.37\\
\hline
LSTMFCNClassifier & 0.41 & 0.5 & 0.3 & 0.37 & -0.13\\
\hline
\end{tabular}
\end{table}

The results of the different classification algorithms are shown in
Table. 2. After completing the analysis and results statistics, four of
the algorithms (ROCKET, DummyClassifier, TimeSeriesSVC, Arsenal) have
the same accuracy of 0.59, precision of 0.59, recall of 1.0, F1-score of
0.74, and Matthews correlation coefficient of 0.0. This is because they
have all classified all of the subjects in the test data set as having
ADHD. The remaining algorithms have a range of accuracy (0.18-0.47),
precision (0.17-0.55), recall (0.1-0.5), F1-scores (1.3-0.52), and
Matthews correlation coefficient (-0.63 - -0.07) but they are all very
low.

\hypertarget{discussion}{%
\section{Discussion}\label{discussion}}

The motor activity data for ADHD people and non-ADHD people appear to be
too similar for the sktime time series algorithms to be able to
accurately classify them. The four algorithms with the highest accuracy,
precision, recall, and F1-score (ROCKET, DummyClassifier, TimeSeriesSVC,
Arsenal) classified all of the test subjects as having ADHD (See Table
2). Their scores are higher only because there were more subjects with
ADHD in the test sample. Based on this observation, their accuracy,
precision, recall, and F1-score would all be zero if there were no
subjects with ADHD in a sample. These would not be valid classification
methods since it appears they would classify all people as having ADHD.
The remaining algorithms (IndividualTDE, KNeighborsTimeSeriesClassifier,
FCNClassifier, LSTMFCMClassifier) all have very low accuracies (range of
0.18-0.47), precision (range of 0.17-0.55), recall (range of 0.1-0.5),
and F1-scores (range of 1.3-0.52). Of these the KNeighborsTimeClassifier
appears to be the best and has the highest accuracy, precision, recall,
F1-score, and Matthews correlation coefficient. These results are too
low to be useful classifiers for ADHD based on motor activity, so this
method of time series preprocessing and algorithms appears not to be the
best method and should not be used for diagnosing ADHD. The analysis
done on the HYPERAKTIV data previously (tsfresh feature engineering and
PCA followed by non-time series classification methods) were able to
achieve more accurate results (\citet{10.1145/3458305.3478454},
\citet{kaur_accurate_2022}). The PCA and SVM based approach used
previously appears to still be the best method
(\citet{kaur_accurate_2022}). It outperformed all of the time series
classification algorithms applied in this paper with an accuracy of
98.43\%, F-measure of 98.42\%, recall of 98.33\%, and precision of
98.56\% (\citet{kaur_accurate_2022}). If motor activity data were going
to be used as a diagnostic method in a clinical setting, PCA processing
them a SVM classification algorithm appears to be the most effective
choice.

It is worth noting that certain procedures in this analysis could be
improved. The preprocessing steps used in this paper such as simple
moving averages and differencing data are generic time series
preprocessing steps used across many different types of time series
data. It is possible that more refined and preferable time series
preprocessing exists for daily motor activity data to correctly identify
the trends in the data that are important. The averages of activity data
over a 24 hour period have a large peak in the middle of the day for
subjects with and without ADHD (See Fig. 1). It is possible if this peak
was correctly removed, there would be differences in motor activity
between the two groups which the machine learning algorithms are able to
identify. Additionally, for all of the algorithms in this paper, the
default hyperparameters were used. If these hyperparameters were
adjusted and optimized, it is possible it could produce more accurate
results. Furthermore, possible confounding variables included in the
demographic HYPERAKTIV data (@) were not included in the time series
classification methods used in this paper due to difficulty including
confounding variables in time series analysis. Preliminary research
revealed limited methods for including confounding variables in time
series classification. Time series machine learning classification
algorithms should not be totally discounted, because it is possible that
improved preprocessing of the data, optimization of the hyperparameters,
and inclusion of confounding variables could lead to improved results.

The data used in the HYPERAKTIV data set is still diagnosed by people
(\citet{10.1145/3458305.3478454}). Even if the diagnoses of ADHD and
other psychiatric disorders were given by two expert, certified
psychologists, this still means the diagnoses are coming from people and
could contain some subjectivity. Using supervised machine learning
methods it would be impossible to get rid of this subjectivity, since it
requires labeled data to make classifications and predictions. To combat
the subjectivity even more, unsupervised machine learning could be
explored as an approach to diagnosing ADHD using motor activity data.
This could remove the need for any human diagnosis, since unsupervised
machine learning does not require labeled data. However, there are
disadvantages to exploring unsupervised machine learning methods such as
unpredictable results and there would be no way to verify if the
conclusions were correct except by an expert, which would reintroduce
subjectivity.

Another factor to consider is that machine learning classification
methods for classifying ADHD based on motor activity is currently not
currently a feasible diagnostic tool, even if the classification methods
are accurate. Applying analysis methods such as feature extraction, PCA,
or time series and non-time series classification algorithms is quite
technologically advanced. For it to become a clinically applicable
diagnostic tool, a tool or service would have to be devised to allow
practitioners to more easily use these methods or have someone else use
these methods for them. If a practitioner wanted to apply the analysis
on motor activity themselves, they would have to use the HYPERAKTIV data
to train the model or collect enough of their own activity data to train
a model. Additionally, people being diagnosed would have to wear the
motor activity monitor for about a week for it to be the same length as
the HYPERAKTIV data, which is longer than other diagnoses methods used
and may be a complication. The actigraph is worn at home, which could
introduce measurement errors if subjects dont' use it properly. For
classifying ADHD from motor activity data to become usable as a clinical
diagnostic tool, there are first obstacles to overcome.

\hypertarget{future-directions}{%
\section{Future Directions}\label{future-directions}}

There are many future directions that research classifying ADHD with
motor activity data (both with the HYPERAKTIV dataset and beyond) could
take. The sample size of the HYPERAKTIV data set is small, with 103
participants and 85 participants with motor activity data. Another
dataset of motor activity could be created with a larger sample size to
reproduce the results and investigate if they are reproducible. The
HYPERAKTIV motor activity data has only been analyzed alone and never in
conjunction with the heart rate data (\citet{10.1145/3458305.3478454},
\citet{kaur_accurate_2022}). It would be interesting to analyze both of
these datasets together, along with the demographic information of the
dataset, and see how it impacts the results. The HYPERAKTIV dataset only
contains data from adults (age 17-67). A possible future direction for
study could be to record motor activity and heart rate data for children
with and without ADHD and then analyze the data to see how it compares
to the adults data and if similar classification methods can be used to
predict the presence of ADHD using the activity data from children.
Finally, making the best classification methods accessible to
practitioners could be another direction of research. Most practitioners
do not have the skill needed to apply any of the machine learning
methods, so to use motor activity data as a diagnostic tool in a
clinical setting a tool for practitioners would have to be produced.

\hypertarget{personal-reflection}{%
\section{Personal Reflection}\label{personal-reflection}}

\textbf{TODO} {[}What was your original research question? Why did you
have to deviate? What ideal dataset would be recommend collecting? How
was this process of developing a research paper? 250 words{]}

\hypertarget{code-availability}{%
\section{Code availability}\label{code-availability}}

All analysis code for this article is available at:
\url{https://github.com/i-m-foster/sds300np-ireneFoster}

\hypertarget{acknowledgements}{%
\section{Acknowledgements}\label{acknowledgements}}

Thank you to my instructor Rosie Dutt and my classmates for their
knowledge, insight, and support.

% %%%%%%%%%%%%%%%%%%%%%%%%%%%%%%%%%%%%%%%%%%
% %% optional
% \supplementary{The following are available online at www.mdpi.com/link, Figure S1: title, Table S1: title, Video S1: title.}
%
% % Only for the journal Methods and Protocols:
% % If you wish to submit a video article, please do so with any other supplementary material.
% % \supplementary{The following are available at www.mdpi.com/link: Figure S1: title, Table S1: title, Video S1: title. A supporting video article is available at doi: link.}

\vspace{6pt}

%%%%%%%%%%%%%%%%%%%%%%%%%%%%%%%%%%%%%%%%%%

%%%%%%%%%%%%%%%%%%%%%%%%%%%%%%%%%%%%%%%%%%

%%%%%%%%%%%%%%%%%%%%%%%%%%%%%%%%%%%%%%%%%%
\conflictsofinterest{The authors declare no conflict of interest.}

%%%%%%%%%%%%%%%%%%%%%%%%%%%%%%%%%%%%%%%%%%
%% optional


%%%%%%%%%%%%%%%%%%%%%%%%%%%%%%%%%%%%%%%%%%
% Citations and References in Supplementary files are permitted provided that they also appear in the reference list here.

%=====================================
% References, variant A: internal bibliography
%=====================================
%\reftitle{References}
%\begin{thebibliography}{999}
% Reference 1
%\bibitem[Author1(year)]{ref-journal}
%Author1, T. The title of the cited article. {\em Journal Abbreviation} {\bf 2008}, {\em 10}, 142--149.
% Reference 2
%\bibitem[Author2(year)]{ref-book}
%Author2, L. The title of the cited contribution. In {\em The Book Title}; Editor1, F., Editor2, A., Eds.; Publishing House: City, Country, 2007; pp. 32--58.
%\end{thebibliography}

% The following MDPI journals use author-date citation: Arts, Econometrics, Economies, Genealogy, Humanities, IJFS, JRFM, Laws, Religions, Risks, Social Sciences. For those journals, please follow the formatting guidelines on http://www.mdpi.com/authors/references
% To cite two works by the same author: \citeauthor{ref-journal-1a} (\citeyear{ref-journal-1a}, \citeyear{ref-journal-1b}). This produces: Whittaker (1967, 1975)
% To cite two works by the same author with specific pages: \citeauthor{ref-journal-3a} (\citeyear{ref-journal-3a}, p. 328; \citeyear{ref-journal-3b}, p.475). This produces: Wong (1999, p. 328; 2000, p. 475)

%=====================================
% References, variant B: external bibliography
%=====================================
\reftitle{References}
\externalbibliography{yes}
\bibliography{mybibfile.bib}

%%%%%%%%%%%%%%%%%%%%%%%%%%%%%%%%%%%%%%%%%%
%% optional

%% for journal Sci
%\reviewreports{\\
%Reviewer 1 comments and authors’ response\\
%Reviewer 2 comments and authors’ response\\
%Reviewer 3 comments and authors’ response
%}

%%%%%%%%%%%%%%%%%%%%%%%%%%%%%%%%%%%%%%%%%%


\end{document}
